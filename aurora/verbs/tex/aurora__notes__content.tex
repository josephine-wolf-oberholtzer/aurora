\begin{multicols}{2}

\section{Instrumentation}
\emph{mbrsi/aurora} has the following instrumentation:\\
\begin{itemize}
	\item[-] 12 violins
	\item[-] 4 violas
	\item[-] 4 cellos
	\item[-] 2 contrabasses
\end{itemize}

\section{Bowing}

\subsection{Bow Positions}

Bow positions from \emph{sul ponticello} to {sul tasto} are indicated in 3-line tablature fragments above the main staff:\\

\begin{figurehere}
\lilypondfile[notime,staffsize=16]{notes__positions.ly}
\end{figurehere}

\subsection{Overpressure}

Overpressure is indicated by a black box and bracket above the bowing-staff:\\

\begin{figurehere}
\lilypondfile[notime,staffsize=16]{notes__overpressure.ly}
\end{figurehere}

\subsection{Circular Bowing / Ponticello-Tasto Tremoli}

Zigzags on the bowing-staff indicate rapid circular bowing (essentially a tremolo from \emph{sul tasto} to \emph{sul ponticello}):\\

\begin{figurehere}
\lilypondfile[notime,staffsize=16]{notes__circular.ly}
\end{figurehere}

\subsection{Jete / Spiccato}

Dotted lines on the bowing-staff indicate a \emph{jéte} or similarly bounced bow:\\

\begin{figurehere}
\lilypondfile[notime,staffsize=16]{notes__jete.ly}
\end{figurehere}

\section{Glissandi}

\subsection{Normal Glissandi}

Two types of glissandi are prescribed.  The first, with a straight line, is to be played as expected:\\

\begin{figurehere}
\lilypondfile[notime,staffsize=16]{notes__glissando.ly}
\end{figurehere}

\subsection{Oscillations}

The second, with a zigzag-line, indicates a glissandi with a very, very wide vibrato, of at least a few semitones:\\

\begin{figurehere}
\lilypondfile[notime,staffsize=16]{notes__oscillation.ly}
\end{figurehere}

\end{multicols}